\documentclass[letterpaper, 10pt, onecolumn, draftclsnofoot]{IEEEtran}
\usepackage{titling}                      
\usepackage{geometry}
\usepackage{enumerate}
\geometry{margin=.75in}
\renewcommand{\familydefault}{\rmdefault}

\title{Problem Statement for Marker Based AR Sandbox Improvements}
\author{Gray, McKenzie. Spencer, Jonah. Sunderman, Adam\\Group 56\\Senior Software Engineering Project\\Fall 2018}
\date{\today}

\begin{document}

\maketitle
\begin{abstract}
Augmented reality sandboxes (AR Sandbox) have become increasingly popular in recent years with topographical image mapping and water flow simulations. The basic concept behind an AR sandbox is simple. The unit is comprised of a physical box of sand, video projector, depth sensor, and computer. As users interact with the sandbox and shape the sand within they will also manipulate in real time an image that is projected on the sand. Despite this increasing popularity AR sandboxes still have a great deal of untapped potential.
\\
This document outlines a plan to use the augmented reality (AR) sandbox at Oregon State University as a means to improve teaching and planning methods in the fields of construction and civil engineering. The proposed changes will add new functionality in the form of a marker-based system for adding various interactive elements to a sandbox. These elements will be used for traffic simulations and the general construction of an image to be displayed on the sand.
\end{abstract}
\newpage

\section{Problem Description} 
 Computer simulations are an effective tool in planning and teaching. Using simulations engineers and scientists can test scenarios in simulated environments without any real-world risk. This technology also provides an opportunity for students to learn in a practical, hands on way without having to leave a classroom. The problem with most computer simulations though is that they require a software engineer to design and setup. Our goal is to reduce this dependency by creating a model for building working computer simulations using physical objects and augmented reality.
\\
Currently, the college of Civil and Construction Engineering's (CCE) Augmented Reality Sandbox (AR Sandbox) is incomplete. Civil engineering professors want to be able to use the AR Sandbox as a traffic simulation environment that can be built and altered using marker-based object placement. Markers would be physical objects placed in the sandbox that are uniquely identifiable by current or additional hardware. The Markers will signal the system to project images of arbitrary objects, such as buildings or street lights, into the sandbox. These objects would then be added to a digital scene on the control computer where the simulation would run its logic.
\\
This feature is the main purpose for this project, but problems arise with the current implementation of the software. The software is hard to set up, the calibration for the depth sensor is nearly non-existent, and the projected elevation images do not display contour lines. We will need to increase usability by refining the current software to meet these needs before moving on to the new features.

\section{Proposed Solution}
For our primary goal, creating a traffic simulation for use in the classroom, we will be expanding upon a terrain mesh generated by the AR Sandbox in Unity Game Engine. This mesh is formed with the current sand height information the AR Sandbox’s depth camera reads in real-time. On this mesh, we will impose pre-made road patterns and run analysis of this road network with a pre-built traffic simulator. We can build the scenes and run the simulations using Unity then project a top-down view of this simulation onto the sandbox. Unity makes it simple to create 3D scenes and has a built-in physics engine that will be useful for simulations.
\\
To implement the AR Marker functionality, we will use the software development kit Vuforia. Vuforia uses computer vision to track and identify real-world objects. It's integrated well with Unity and provides an easy, powerful interface for digitally re-creating a 3D object through object recognition. Objects, when placed in the sandbox and recognized, will match a pre-defined Unity asset which will be drawn to a scene a where the marker a placed.
\\
To fix the usability problems of the current system we will need to conduct research on how the system is used. We know how the system works but we don't know how civil engineering professors and students use it currently. Once we have this information we can begin to design a new user interface that will be easier to use and tailored to the people using it.

\section{Performance Metrics}
The primary metric by which this project will be evaluated is the creation of a working traffic simulation. As such there are several performance metrics we must achieve to ensure this goal is met. 
\\
\begin{enumerate}
	\item{}
	We need to deliver a programable and expandable marker system that can support a wide variety of marker types.
	\item{}
	We need to deliver a system capable of translating markers into proper models that can be displayed on the surface of the sand.
	\item{}
	We need to deliver a configurable traffic simulation that is impacted by the placement of markers.
	\item{}
	We need to refine the system to be more useable by civil engineering students.
\end{enumerate}
\end{document}