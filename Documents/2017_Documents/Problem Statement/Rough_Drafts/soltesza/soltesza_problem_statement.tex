\documentclass[onecolumn, draftclsnofoot,10pt, compsoc]{IEEEtran}
\usepackage{graphicx}
\usepackage{url}
\usepackage{setspace}

\usepackage{geometry}
\geometry{textheight=9.5in, textwidth=7in}

% 1. Fill in these details
\def \CapstoneTeamName{			AR Sandbox for Construction Planning}
\def \CapstoneTeamNumber{		44}
\def \GroupMemberOne{			Andrew Soltesz}
\def \GroupMemberTwo{			Raja Petroff}
\def \GroupMemberThree{			Mark Sprouse}
\def \CapstoneProjectName{		AR Sandbox for Construction Planning}
\def \CapstoneSponsorCompany{	Oregon State University}
\def \CapstoneSponsorPerson{		Dr. Joseph Louis}

% 2. Uncomment the appropriate line below so that the document type works
\def \DocType{		Problem Statement
				%Requirements Document
				%Technology Review
				%Design Document
				%Progress Report
				}
			
\newcommand{\NameSigPair}[1]{\par
\makebox[2.75in][r]{#1} \hfil 	\makebox[3.25in]{\makebox[2.25in]{\hrulefill} \hfill		\makebox[.75in]{\hrulefill}}
\par\vspace{-12pt} \textit{\tiny\noindent
\makebox[2.75in]{} \hfil		\makebox[3.25in]{\makebox[2.25in][r]{Signature} \hfill	\makebox[.75in][r]{Date}}}}
% 3. If the document is not to be signed, uncomment the RENEWcommand below
%\renewcommand{\NameSigPair}[1]{#1}

%%%%%%%%%%%%%%%%%%%%%%%%%%%%%%%%%%%%%%%
\begin{document}
\begin{titlepage}
    \pagenumbering{gobble}
    \begin{singlespace}
    	%\includegraphics[height=4cm]{coe_v_spot1}
        \hfill 
        % 4. If you have a logo, use this includegraphics command to put it on the coversheet.
        %\includegraphics[height=4cm]{CompanyLogo}   
        \par\vspace{.2in}
        \centering
        \scshape{
            \huge CS Capstone \DocType \par
            {\large\today}\par
            \vspace{.5in}
            \textbf{\Huge\CapstoneProjectName}\par
            \vfill
            {\large Prepared for}\par
            \Huge \CapstoneSponsorCompany\par
            \vspace{5pt}
            {\Large\NameSigPair{\CapstoneSponsorPerson}\par}
            {\large Prepared by }\par
            Group\CapstoneTeamNumber\par
            % 5. comment out the line below this one if you do not wish to name your team
            \CapstoneTeamName\par 
            \vspace{5pt}
            {\Large
                \NameSigPair{\GroupMemberOne}\par
                \NameSigPair{\GroupMemberTwo}\par
                \NameSigPair{\GroupMemberThree}\par
            }
            \vspace{20pt}
        }
        \begin{abstract}
        % 6. Fill in your abstract    
        	This document describes the concept of an augmented reality sandbox and how we want to leverage this technology to create a tool for collaborative highway planning. 
            Additionally, an outline of our proposed implementation is discussed, detailing our intention to use prebuilt hardware to capture height data from our sandbox, which will interface with the Unity game engine through a prebuilt plugin. 
            Potential challenges and sticking points are introduced, as well as speculative solutions to these issues. 
            Finally, the metrics by which we will gauge the efficacy of this solution are discussed, such as time and functionality requirements. 
            
        \end{abstract}     
    \end{singlespace}
\end{titlepage}
\newpage
\pagenumbering{arabic}
\tableofcontents
% 7. uncomment this (if applicable). Consider adding a page break.
%\listoffigures
%\listoftables
\clearpage

% 8. now you write!
\section{Project Description}
The purpose of this project is to expand on the concept of an augmented reality (AR) sandbox, which projects digital information on a tactile physical surface.
This is achieved by positioning some form of 3d camera over the sandbox to capture height data and make a digital replica of the surface of the sand.
Various digital effects can be applied to this 3D representation, such as contour lines and water simulation.
The digital representation is then rendered from a top down view, and the image is sent to a projector which projects this image onto the sand.
Currently, implementations of this technology are limited in their practicality and exist for the most part as a novelty, rather than a functional visualization tool.
With this project we hope to use this medium to create a useful tool for collaborative highway planning. 
To accomplish this, we need to implement a system for drawing and editing roads on the sandbox's topology. 
Additionally, we need to design a method for user control over the creation and editing of these roads without a mouse and keyboard.

\section{Proposed Solution}
In order to implement a solution to this problem, we will use the Unity game engine to build and deploy our software.
Unity has the advantage of being relatively easy to learn, as well as having the option to deploy on multiple platforms.
Additionally, the software is free.
We will use the Microsoft Kinect to read height data from the sandbox, which will interface with Unity using the Kinect for Unity plugin.
The Kinect has the advantage of being a low cost prebuilt 3D camera with an already well established community of developers.
A projector will be used to display the resulting image on the sand.
The specifications of the projector aren't as important, as long as it has a close focus distance of roughly a meter or less so as to be able to project onto the sand while keeping the overall height of the sandbox reasonable.

The first step will be to obtain height data from the sandbox and replicate the surface of the sand within Unity.
Getting the height data from the Kinect should be relatively straight forward as the Kinect for Unity plugin should handle most of the computation here.
Generating the surface of the sandbox in 3D will most likely pose more of a challenge as we will have to figure out how to sample the Kinect's height data and create a dynamic mesh within Unity.
Then, we will be able to begin using this mesh to draw the various features we want displayed on the sand, such as color-coded height information and contour lines.
At this point we will have a functioning AR sandbox, and can begin implementing roadway editing and alignment tools, as well as designing a way to allow users to control these tools organically without the aid of a mouse and keyboard.

The most complicated part of this project will most likely be implementing the basic sandbox functionality such as reading height data from the Kinect and constructing an accurate three dimensional mesh within the game engine. 
Building the sandbox itself may also pose some unforeseen challenges, however this portion of the project has been well documented online and should be relatively straight forward since all major hardware components can be purchased off the shelf.

\section{Performance Metrics}
In order to judge the completeness of this project, a number of metrics have been established.
The first and most basic requirement is a working AR sandbox onto which information can be projected.
This entails having the box of sand itself, a method of mounting the Kinect and projector, and a computer equipped with a capable video card.
An equally rudimentary requirement is a piece of software capable of interpreting the Kinect's data and projecting an image onto the sand.
Projections should be precise and line up with the physical geometry of the sand, plus or minus one projected pixel.
This can be tested by setting an object with a flat top and vertical sides such as a cube on the sand. If the projection lines up with the edges of the cube, the projector is properly calibrated.
Digital information should begin updating in less than a second so as to allow quick, organic user interaction.
In previous implementations, the projection would update only a small part of the sandbox at a time as the user sculpts the sand. This is an acceptable and actually preferable solution as it decreases any perceived delay.
Additionally, the sandbox should have the ability to display and edit roadway alignments with the help of a hand-held device.
Ideally, the sandbox should function as an independent device, rather than a computer peripheral and should not require a mouse and keyboard to use any of its features. 
The sandbox should be efficient enough to run within the aforementioned latency constraints with modern consumer grade graphics hardware, such as the Nvidia GTX 1070 or AMD Radeon RX Vega. 

\end{document}