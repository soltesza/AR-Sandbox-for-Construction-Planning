\documentclass[onecolumn, draftclsnofoot,10pt, compsoc]{IEEEtran}
\usepackage{graphicx}
\usepackage{url}
\usepackage{setspace}

\usepackage{geometry}
\geometry{textheight=9.5in, textwidth=7in}

% 1. Fill in these details
\def \CapstoneTeamName{		The Cleverly Named Team}
\def \CapstoneTeamNumber{		44}
\def \GroupMemberOne{			Raja Petroff}
\def \GroupMemberTwo{			Andrew Soltesz}
\def \GroupMemberThree{			Mark Sprouse}
\def \CapstoneProjectName{		AR Sandbox for Construction Planning}
\def \CapstoneSponsorCompany{	Civil \& Construction Engineering}
\def \CapstoneSponsorPerson{		Dr. Joseph Louis}
\title{Group 44 Problem Statement}

% 2. Uncomment the appropriate line below so that the document type works
\def \DocType{		Problem Statement
				%Requirements Document
				%Technology Review
				%Design Document
				%Progress Report
				}
			
\newcommand{\NameSigPair}[1]{\par
\makebox[2.75in][r]{#1} \hfil 	\makebox[3.25in]{\makebox[2.25in]{\hrulefill} \hfill		\makebox[.75in]{\hrulefill}}
\par\vspace{-12pt} \textit{\tiny\noindent
\makebox[2.75in]{} \hfil		\makebox[3.25in]{\makebox[2.25in][r]{Signature} \hfill	\makebox[.75in][r]{Date}}}}
% 3. If the document is not to be signed, uncomment the RENEWcommand below
\renewcommand{\NameSigPair}[1]{#1}

%%%%%%%%%%%%%%%%%%%%%%%%%%%%%%%%%%%%%%%
\begin{document}
\begin{titlepage}
    \pagenumbering{gobble}
    \begin{singlespace}
    	%\includegraphics[height=4cm]{coe_v_spot1}
        \hfill 
        % 4. If you have a logo, use this includegraphics command to put it on the coversheet.
        %\includegraphics[height=4cm]{CompanyLogo}   
        \par\vspace{.2in}
        \centering
        \scshape{
            \huge CS Capstone \DocType \par
            {\large\today}\par
            \vspace{.5in}
            \textbf{\Huge\CapstoneProjectName}\par
            \vfill
            {\large Prepared for}\par
            \Huge \CapstoneSponsorCompany\par
            \vspace{5pt}
            {\Large\NameSigPair{\CapstoneSponsorPerson}\par}
            {\large Prepared by }\par
            Group\CapstoneTeamNumber\par
            % 5. comment out the line below this one if you do not wish to name your team
            %\CapstoneTeamName\par 
            \vspace{5pt}
            {\Large
                \NameSigPair{\GroupMemberOne}\par
                \NameSigPair{\GroupMemberTwo}\par
                \NameSigPair{\GroupMemberThree}\par
            }
            \vspace{20pt}
        }
        \begin{abstract}
        % 6. Fill in your abstract    
        	Current techniques for building roads, bridges, and other civil engineering projects utilize two dimensional blueprints and three dimensional computer aided design.
			However, these modern methods can be improved upon by using more interactive and collaborative interfaces.
			Our project consists of creating an augmented reality sandbox, which will consist of a depth sensor and display projector, a novel software solution utilizing a game engine to project 3D graphics on the sand.
			The use of the sandbox, in particular, will allow for more tactile user interaction with construction or road data.
        \end{abstract}     
    \end{singlespace}
\end{titlepage}
\newpage
\pagenumbering{arabic}
\tableofcontents
% 7. uncomment this (if applicable). Consider adding a page break.
%\listoffigures
%\listoftables
\clearpage

% 8. now you write!
\section{Problem Description}
The goal of this project is to create a more innovative and interactive method of visualizing construction and highway planning.
Currently, highway and construction planning uses a static drawing or three dimensional design which is iteratively reworked until sufficient.
Creating an augmented reality interface will allow for better collaboration and a more tactile experience when planning for construction, road building, or many other forms of civil engineering.
This will allow for near real-time feedback and an easier way to visualize the planning data for all phases of construction and design.
\par
Collaboration among members of a design committee or different groups (such as state transportation agencies, construction contractors, design agencies, or the public at large) will be greatly enhanced by this augmented reality interface.
It will ultimately allow those who are not usually involved in the design or planning process to take part and give valid input into a project that may affect them.
\par
Eventually, this project can be expanded to include additional real world  problems, such as earthwork planning and disaster recovery.
Especially when recovering after a disaster, it is important to plan ahead. Likewise, before an earthworks project can go forward, it must be planned out and simulated.
This sandbox project will allow for interactive physical simulations that users can manipulate with their own hands.

\section{Proposed Solution}
We aim to create a self contained, interactive, and collaborative device, therefore our project has elements of both hardware and software.
Our solution is a sandbox with an integrated depth sensor and display projector; a novel piece of software that would project 3D graphics onto the sand; and, optionally, a pointing device that can be used to interact with the software interface independent of the physical sandbox itself.
\par
For our project we will most likely be using off the shelf parts.
The depth sensor for example, will probably take the form of a Microsoft Kinect.
The display projector will be used to overlay the graphics and user interface over the sand.
The software used for the 3D graphics will be a game engine, specifically something free such as Unity or the Unreal Engine.
The benefit of using one of the aforementioned engines is that critical portions of the program such as interfacing with the depth sensor are handled by prebuilt plugins.
The pointing device used to interact with the software could be either a custom piece of hardware or something off the shelf such as a Wii Remote.

\section{Performance Metrics}
The performance metrics for our project will primarily consist of a working prototype or proof of concept augmented reality sandbox.
This entails creating both the software program used for construction and road planning, as well as the sandbox itself.
This will depend on successfully implementing a 3D graphics engine, either Unity or the Unreal Engine, with our own software interface.
The software should be capable of projecting a computer generated image on the sandbox that aligns with the physical geometry within a single projected pixel.
At the very least, the software should display the path of a single road on the sandbox, along with a dynamic cut and fill table for that road that changes as the topography of the sandbox is modified.
Digital information should begin updating in less than a second so as to allow quick, organic user interaction.
In previous implementations, the projection would update only a small part of the sandbox at a time as the user sculpts the sand. 
This is an acceptable and actually preferable solution as it decreases any perceived delay.
Along with the software, we will need to utilize hardware such as a Microsoft Kinect, or some sort of depth sensor, as well as a projector in order to display imagery on the sand.
\par
Lastly, if we manage to complete the aforementioned requirements before the project deadline, we will attempt to implement some of the following optional features.
The first feature is a pointing device that can be used to interact with the sandbox.
This device can take the form of either a custom piece of hardware or something off the shelf such as a Wii Remote.
Alternatively, we could implement a suite of additional design tools for the sandbox such as cut and fill analysis for a user-inputted road segment, or for solving other civil engineering problems such as stopping sight distance.



\end{document}
