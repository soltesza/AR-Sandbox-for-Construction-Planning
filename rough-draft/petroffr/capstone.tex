\documentclass[onecolumn, draftclsnofoot,10pt, compsoc]{IEEEtran}
\usepackage{graphicx}
\usepackage{url}
\usepackage{setspace}

\usepackage{geometry}
\geometry{textheight=9.5in, textwidth=7in}

% 1. Fill in these details
\def \CapstoneTeamName{		The Cleverly Named Team}
\def \CapstoneTeamNumber{		44}
\def \GroupMemberOne{			Raja Petroff}
\def \GroupMemberTwo{			Andrew Soltesz}
\def \GroupMemberThree{			Mark Sprouse}
\def \CapstoneProjectName{		AR Sandbox for Construction Planning}
\def \CapstoneSponsorCompany{	Civil \& Construction Engineering}
\def \CapstoneSponsorPerson{		Joseph Louis}

% 2. Uncomment the appropriate line below so that the document type works
\def \DocType{		Problem Statement
				%Requirements Document
				%Technology Review
				%Design Document
				%Progress Report
				}
			
\newcommand{\NameSigPair}[1]{\par
\makebox[2.75in][r]{#1} \hfil 	\makebox[3.25in]{\makebox[2.25in]{\hrulefill} \hfill		\makebox[.75in]{\hrulefill}}
\par\vspace{-12pt} \textit{\tiny\noindent
\makebox[2.75in]{} \hfil		\makebox[3.25in]{\makebox[2.25in][r]{Signature} \hfill	\makebox[.75in][r]{Date}}}}
% 3. If the document is not to be signed, uncomment the RENEWcommand below
\renewcommand{\NameSigPair}[1]{#1}

%%%%%%%%%%%%%%%%%%%%%%%%%%%%%%%%%%%%%%%
\begin{document}
\begin{titlepage}
    \pagenumbering{gobble}
    \begin{singlespace}
    	\includegraphics[height=4cm]{coe_v_spot1}
        \hfill 
        % 4. If you have a logo, use this includegraphics command to put it on the coversheet.
        %\includegraphics[height=4cm]{CompanyLogo}   
        \par\vspace{.2in}
        \centering
        \scshape{
            \huge CS Capstone \DocType \par
            {\large\today}\par
            \vspace{.5in}
            \textbf{\Huge\CapstoneProjectName}\par
            \vfill
            {\large Prepared for}\par
            \Huge \CapstoneSponsorCompany\par
            \vspace{5pt}
            {\Large\NameSigPair{\CapstoneSponsorPerson}\par}
            {\large Prepared by }\par
            Group\CapstoneTeamNumber\par
            % 5. comment out the line below this one if you do not wish to name your team
            %\CapstoneTeamName\par 
            \vspace{5pt}
            {\Large
                \NameSigPair{\GroupMemberOne}\par
                \NameSigPair{\GroupMemberTwo}\par
                \NameSigPair{\GroupMemberThree}\par
            }
            \vspace{20pt}
        }
        \begin{abstract}
        % 6. Fill in your abstract    
        	Current techniques for building roads, bridges, and other civil engineering projects utilize two dimensional blueprints and three dimensional computer aided design.
			However, these modern methods of highway and construction planning can be improved upon using more interactive and collaborative interfaces.
			Therefore, our project consists of creating an augmented reality sandbox.
			This sandbox would consist of a depth sensor and display projector, a novel software solution utilizing a game engine to project 3D graphics on the sand, and some sort of handheld mouse or pointer device to interact with elements of the user interface.
			The use of a sandbox in particular will allow for more tactile user interaction with construction or road data.
        \end{abstract}     
    \end{singlespace}
\end{titlepage}
\newpage
\pagenumbering{arabic}
\tableofcontents
% 7. uncomment this (if applicable). Consider adding a page break.
%\listoffigures
%\listoftables
\clearpage

% 8. now you write!
\section{Problem Description}
The problem we are trying to solve is creating a more innovative and interactive method of visualizing construction and highway planning.
Currently, highway and construction planning uses a static drawing or three dimensional design which is iteratively reworked until sufficient.
Indeed, it can be argued that creating an augmented reality interface would allow for better collaboration and a more tactile experience when planning for construction, road building, or any other form of civil engineering.
This would allow for instant feedback and an easier way to visualize planning data for all phases of construction and design.
\par
Collaboration among members of a design committee or among different groups, such as state transportation agencies, construction contractors, design agencies, and even the public at large would be greatly enhanced by this augmented reality interface.
It would ultimately allow those who are not usually involved in the design or planning process to take part and give valid input into a project that may affect them.
\par
This project can be ultimately expanded to include other real world  problems such as earthwork planning and disaster recovery.
Especially when recovering after a disaster it is important to plan ahead. Likewise, before an earthworks project can go forward, it must be planned out and simulated.
This sandbox project would allow for interactive physical simulations that users can manipulate with their own hands.

\section{Proposed Solution}
We aim to create an interactive and collaborative device, therefore our project has elements of both hardware and software.
Our solution is a sandbox with an integrated depth sensor and display projector; a novel piece of software that would project 3D graphics onto the sand; and a pointing device that can be used to interact with the software interface independent of the physical sandbox itself.
\par
For our project we will most likely be using off the shelf parts.
The depth sensor for example, will probably have to be a Microsoft Kinect.
The display projector will be used to overlay the graphics and user interface over the sand.
The software used for the 3D graphics will be a game engine, probably something free such as Unity or the Unreal Engine.
The pointing device used to interact with the software could be either a custom piece of hardware or something off the shelf such as a Wii Remote.

\section{Performance Metrics}
The performance metrics for our project will primarily consist of a working prototype or proof of concept augmented reality sandbox.
The deliverables in this case will consist both of the software program used for construction and road planning, as well as the hardware pointer device used to interact with the software.
This will depend on successfully implementing the 3D graphics engine, either Unity or the Unreal Engine, with our own software interface.
Along with the software, we will need to utilize hardware such as a Microsoft Kinect, or some sort of depth sensor, and a Wii Remote or other kind of pointing device.
\par
Lastly, if the scope of this project turns out to be more than our team can handle, we may end up implementing only the software application and drop the user interface pointing device from our requirements.
The most important part of the project will most likely be the construction and roadway planning software application.
We will be able to discuss this further with our client if it turns out that we need to narrow the scope of our project.


\end{document}